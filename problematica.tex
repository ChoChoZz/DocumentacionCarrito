\chapter{Propuesta}

	\section{Problemática}
	Las ventas del establecimiento han disminuido de un año a la fecha, es necesario mejorar las ventas para tener mejores ingresos y que el negocio pueda ser rentable.

	\section{Causas}
	Para realizar una compra un cliente tiene que ir al establecimiento a visualizar la mercancía existente, esperar a que un vendedor de piso lo atienda y que un cajero le reciba su pago para realizar su compra.\\
	 Las limitantes de dicho mecanismo son:
	 \begin{itemize}
	 	\item Solo se pueden realizar compras mientras la tienda se encuentre abierta.
	 	\item Para realizar una compra tiene que haber un vendedor de piso disponible que le proporcione la mercancia a dicho cliente.
	 	\item Es necesario ir con un cajero que se encuentre desocupado para que le reciba su pago al cliente.
	 \end{itemize}
	
	\section{Impacto}
	Cuando se implemente el sistema se pretende mejorar los ventas del establecimiento, abarcar otro tipo de mercado, crear un mecanismo de venta automatizado y tener un mejor control del inventario.
	
	
	\section{Objetivo}
		Crear una tienda virtual para mejorar las ventas del cliente, teniendo mejores ingresos.
		\subsubsection{Objetivos específicos}
			\begin{itemize}
				\item Construir una tienda que se encuentre disponible pare ventas los trescientos sesenta y cinco días del a\~no las veinte cuatro horas.
				\item Crear un mecanismo que atienda a los clientes de forma muy similar a la actual.
			\end{itemize}
	
	\section{Justificación}
	El servicio actual de venta suele ser un mecanismo engorroso el cual en muchas de las ocasiones es necesario invertirle un par de horas para adquirir un articulo, por ende es necesario crear el sistema para hacer del establecimiento un negocio rentable, esto gracias a que hoy en día gran parte de los consumidores prefieren realizar compras vía internet, evitando perder el tiempo teniendo que ir al establecimiento en cuestión y esperar a que alguien los atienda.\\ \\
	