\chapter{Introducci\'on}

\section{Alcance}
En el presente documento se muestra la documentación que se requiere para crear una tienda virtual, el documento refleja el análisis necesario para llevar a cabo la programación de esta.\\ \\ Las secciones para definir el análisis son:

\begin{itemize}
    \item[•] \textbf{Propuesta:} Muestra la problemática y objetivos que se definen en base al análisis para mitigar las causas de estos.
             \begin{itemize}
             	\item[•] Problemática.
             	\item[•] Causas.
             	\item[•] Impacto.
             	\item[•] Objetivos.
             	\item[•] Justificación.
             \end{itemize}
	\item[•] \textbf{Procesos:} Definirá los procesos necesarios para manejar el negocio y permitir una correcta funcionalidad, en base a los objetivos planteados.
	\item[•] \textbf{Documento de Definición de Requerimientos(DDR):} Listado de los requerimientos funcionales y no funcionales que tendrá el sistema.
	\item[•] \textbf{Casos de Uso:} Es una descripción de los pasos que se llevan a cabo para realizar alguno de los procesos  del sistema, es la parte principal del documento ya que permite que la programación e implementación de lo desarrollado resulte sencilla.
	\item[•] \textbf{Catálogo de mensajes:} Son los mensajes que aparecen cuando alguno de los actores realiza erróneamente alguna función, para confirmar el comienzo de alguna operación o para informarle que la tarea que deseaba realizar se llevo a término con éxito. Existe un mensaje para cada función dentro de cada proceso, por ello la necesidad de contar con un catálogo que organice cada uno de estos.
	\item[•] \textbf{Reglas del Negocio:} Describe las políticas, normas, definiciones, restricciones que rigen la organización y funcionamiento que tendr\'a la tienda virtual.
	\item[•] \textbf{Perfiles:} Describe los perfiles que tendrá el sistema.
	\item[•] \textbf{Glosario:} Apartado para definir los conceptos necesarios para tener una mejor comprensión de lo que se esta abordando.
	\item[•] \textbf{Modelo Conceptual:} Describe las entidades con sus características favoreciendo la comprensi\'on del usuario y los requisitos del software.
\end{itemize}