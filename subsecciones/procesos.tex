\chapter{Procesos}
\textbf{\Large Introducci\'on \\}
\textit{En el presente documento se describen los procesos que existen en SpaceShop los cuales definen la funcionalidad del sistema. \\ \\}

A continuaci\'on se presenta el listado de los procesos existentes:
\begin{itemize}
	\item \hyperlink{procesoCompra}{Comprar.}
	\item \hyperlink{procesoAtenderPedido}{Atender pedido.}
	\item \hyperlink{procesoRegistrarEntidad}{Registrar entidad.}
	\item \hyperlink{procesoEliminarEntidad}{Eliminar entidad.}
	\item \hyperlink{procesoGestionDeInventario}{Gesti\'on de inventario.}
	\begin{itemize}
		\item \hyperlink{procesoRegistrarProductos}{Registrar productos.}
		\item \hyperlink{procesoEliminarProductos}{Eliminar productos.}	
	\end{itemize}
\end{itemize}

\newpage
\textit{\\}
\textbf{\Large Datos relevantes para el an\'alisis de procesos\\\\}
Los datos que se muestran a continuaci\'on son necesarios para comprender el funcionamiento de los procesos que se describen en el documento.
\textit{\\}

\begin{enumerate}
	\item Resumen del proceso: Es una explicación del proceso a grandes rasgos; ayuda a tener una idea general del flujo de las actividades.
	\item Actores que participan en el proceso: Son las personas o entidades que participan en el proceso.
	\item Diagrama de componentes: Es una representación gráfica del proceso usando la notación de Modelado de Procesos de Negocio (BPMN). En éste se observa el flujo que sigue el
 proceso desde que inicia hasta que termina.
	\item Objetivo general: Establece los alcances y principios del proceso. Responde a las preguntas de ¿Qué?, ¿Dónde?, ¿Cuándo?, ¿Por qué? y ¿Para qué?.
	\item Objetivos particulares: Son objetivos que intervienen en el cumplimiento del objetivo general; inciden en metas indirectas y de un alcance menor que el objetivo general.
	\item Insumos de entrada: Son las condiciones que deben cumplirse para que el proceso pueda dar inicio.
	\item Elementos de entrada: Son todos los productos (documentos y/o datos) que provienen de una fuente externa al proceso pero que son importantes dentro de éste.
	\item Salidas: Son todos los productos (documentos y/o datos) que se generan en el transcurso del proceso.
	\item Clientes o consumidores: Son las personas o entidades que hacen uso del proceso.
	\item Mecanismos de medición: Es la forma de verificar que el proceso es correcto.
	\item Interacciones con otros procesos: Son las relaciones que existen entre procesos.
	\item Tipos de solicitud que ejecuta el proceso: Se refiere a quién solicita el proceso (puede ser el cliente o el administrador, por ejemplo)
	\item Descripción de las actividades: Aquí se describe el funcionamiento o comportamiento de cada actividad del proceso.
\end{enumerate}